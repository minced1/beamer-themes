\PassOptionsToClass{handout}{beamer}
\documentclass[aspectratio=169,usepdftitle=true]{beamer}

\usepackage[T1]{fontenc}
\usepackage[utf8]{luainputenc}
%\usepackage[utf8]{inputenc}
\usepackage{microtype}
\usepackage{unicode-math}
\setmathfont{latinmodern-math.otf}
\usepackage{csquotes}
\usepackage[ngerman]{babel}
\usepackage{lipsum}
\usepackage{listings}

\usetheme{dividing-lines}
\SetColorProfile{159, 166, 2}{249, 224, 176}{137, 164, 75}

\title{Neuronale Netze}
\subtitle{ChatGPT und Co.}
\institute{Carl-Friedrich-Gauß-Gymnasium}
\license{\ccbysa}

\date{\today}

\author{Jasper Gude}
%\email{j.gude@posteo.de}
\outro{Hockenheim, \today}
\license[]{\url{https://creativecommons.org/licenses/by-sa/4.0/}\quad\ccbysa}

% just an example command
\newcommand\twosplit[3][t]{%
\begin{columns}[#1]
\begin{column}{0.475\linewidth}#2\end{column}\hfill
\begin{column}{0.475\linewidth}#3\end{column}
\end{columns}}

\def\typesetBox#1#2#3#4{%
    \pbox{7cm}{{\Large\sbfamily#1}\rlap{\textsuperscript{\color{gray}\cite{#4}}}\\\color{gray}#2\hfill#3}%
}

\colorlet{semigray}{gray!70!darkgray}

\lstset{
    backgroundcolor=\color{btdl@color@black},
    commentstyle=\color{semigray},
    keywordstyle=\bfseries\color{btdl@color@primary},
    numberstyle=\tiny\color{semigray},
    stringstyle=\color{btdl@color@secondary},
    basicstyle=\ttfamily\color{btdl@color@white},
    breakatwhitespace=false,
    breaklines=true,
    captionpos=b,
    keepspaces=true,
    numbers=left,
    numbersep=5pt,
    showspaces=false,
    showstringspaces=false,
    showtabs=false,
    tabsize=2
}

\usetikzlibrary{arrows.meta,calc,backgrounds,shapes.multipart,intersections}
\tikzset{
    bc/.style={circle,draw,fill,#1,text=darkgray,align=center,text width=0.75em},
    brs/.style={rectangle,rounded corners=2.5pt,draw,fill,#1,text=darkgray,align=center},
    br/.style={brs=#1,rectangle split,rectangle split parts=2,draw=btdl@color@background,thick,font=\scriptsize\sffamily},
    br/.default={hl-shade},
    bcsmall/.style={circle,draw,fill,#1,text=darkgray,align=center,inner sep=2pt},
    bc/.default={hl-shade},
    bcsmall/.default={hl-shade},
    t/.style={ultra thick,#1},
    tarrow/.style={-Kite,t=#1},
    dtarrow/.style={Kite-Kite,t=#1}
}

\def\cfl#1{\faFile\llap{\color{#1}\tiny\faCogs\;}}
\def\tagS#1#2{\node[above left=-0.082cm,darkgray,scale=0.65,circle,fill=lgray,inner sep=1pt] at(#1.south east) {\sbfamily#2};}
\def\tagN#1#2{\node[below left=-0.082cm,darkgray,scale=0.65,circle,fill=hl-shade,inner sep=1pt] at(#1.north east) {\sbfamily#2};}

\colorlet{lgray}{lightgray!45!white}
\colorlet{hl-shade}{btdl@color@secondary}
\usepackage[backend=biber,style=alphabetic]{biblatex}
\addbibresource{example.bib}
\begin{document}

\section{Theoretische Grundlagen}
\subsection{Der Algorithmusbegriff}
\begin{frame}{Was ist ein Algorithmus?}
    \begin{definition}[Algorithmus]
        Ein \emph{Algorithmus} ist eine \emph{eindeutige Handlungsvorschrift} zur Lösung eines Problems.
    \end{definition}
    \begin{itemize}
        \item Es existieren verschiedene Darstellungsformen: \begin{itemize}
            \item Textuell (1. Tue dies, 2. Tue das, 3. Falls \(X\)\ldots)
            \item Grafisch (Instruktionsabläufe, Graphen, Grafiken, \ldots)
            \item Pseudocode
        \end{itemize}
        \item Klassische Alltagsbeispiele: (Koch-)Rezepte, Gebrauchsanleitungen, \ldots
    \end{itemize}
\end{frame}

\subsection{Algorithmen analysieren}
\begin{frame}{Algorithmuseigenschaften}
    \begin{itemize}
        \item Zu Beginn gilt es einige Begriffe zu klären: \begin{description}[Elementaroperation]
            \item[Prozess] Die Ausführung der Schritte eines Algorithmus
            \item[Prozessor] Der Ausführende (Mensch, Computer, \ldots)
            \item[Elementaroperation] Eine einzelne, eindeutige Handlung.
        \end{description}
    \end{itemize}
\end{frame}

\begin{frame}[c]{Eigenschaften für die Analyse}
    \twosplit{%
    \begin{block}{Korrektheit}
        Ein Algorithmus ist korrekt, wenn er für jede (definierte) Eingabe, die korrekte Ausgabe erzeugt.
    \end{block}
    \begin{block}{Partielle Korrektheit}
        Wenn der Algorithmus terminiert, ist er korrekt.
    \end{block}
    }{%
    \begin{block}{Robust}
        Ein Algorithmus ist (maximal) robust, wenn er falsche Eingabe erkennen und abfangen kann.
    \end{block}
    \begin{block}{Totale Korrektheit}
        Der Algorithmus ist partiell korrekt und terminiert.
    \end{block}
    }
\end{frame}

\begin{frame}{Eigenschaften für die Analyse}
    \begin{itemize}
        \item Die Begriffe sind zu unterscheiden!
        \item So können bei einem Algorithmus verschiedene Wege zum selben Ziel führen.
        \item (Sinnfreies) Beispiel: Der Algorithmus kann jedes Element eines Arrays quadrieren. Die Reihenfolge wählt er zufällig aus!
        \item Dieser Algorithmus \emph{determiniert}, ist aber \emph{nicht-deterministisch}!
    \end{itemize}
\end{frame}


\section{Another example}
\subsection{This is an example}

\begin{frame}{Mega-Example 1}
    Hello \cite{knuth-fa}!
\end{frame}

\begin{frame}{Mega-Example 2}
    \begin{layout-imageonly}
    \centering
    \begin{tikzpicture}[bc/.default={lgray}]
    \onslide<2->{\node[bc] (0) at (0,1.4) {\faFileText};}
    \onslide<3->{\node[bc] (1) at (-1.25,0) {\!\faLaptop};
    \node[bc] (2) at ( 0,0) {\!\faLaptop};
    \node[bc] (3) at ( 1.25,0) {\!\faLaptop};

    \draw[tarrow=lgray] (0) -- (1);
    \draw[tarrow=lgray] (0) -- (2);
    \draw[tarrow=lgray] (0) -- (3);}

    \onslide<4->{\node[bc=lgray] (a1) at (-1.25,-1.4) {\cfl{lgray}};
    \node[bc=lgray] (a2) at ( 0,-1.4) {\cfl{lgray}};
    \node[bc=lgray] (a3) at ( 1.25,-1.4) {\cfl{lgray}};

    \draw[tarrow=lgray] (1) -- (a1);
    \draw[tarrow=lgray] (2) -- (a2);
    \draw[tarrow=lgray] (3) -- (a3);}

    \onslide<5->{\path (a1) -- (a2) node[btdl@color@primary,pos=0.5] {$=$};
    \path (a2) -- (a3) node[btdl@color@primary,pos=0.5] {$=$};}
    \onslide<2->{\node[below=.15cm] at (current bounding box.south) {Reproducible};}

\end{tikzpicture}
\end{layout-imageonly}
\end{frame}

\begin{frame}{Mega-Example 3}
    \begin{layout-imageonly}
    \centering
    \begin{tikzpicture}[bc/.default={lgray}]
    \node[bc] (0) at (0,1.4) {\faFileText};
    \node[bc] (1) at (-1.25,0) {\!\faLaptop};
    \node[bc] (2) at ( 0,0) {\!\faLaptop};
    \node[bc] (3) at ( 1.25,0) {\!\faLaptop};

    \draw[tarrow=lgray] (0) -- (1);
    \draw[tarrow=lgray] (0) -- (2);
    \draw[tarrow=lgray] (0) -- (3);

    \node[bc=lgray] (a1) at (-1.25,-1.4) {\cfl{lgray}};
    \node[bc=lgray] (a2) at ( 0,-1.4) {\cfl{lgray}};
    \node[bc=lgray] (a3) at ( 1.25,-1.4) {\cfl{lgray}};

    \draw[tarrow=lgray] (1) -- (a1);
    \draw[tarrow=lgray] (2) -- (a2);
    \draw[tarrow=lgray] (3) -- (a3);

   \path (a1) -- (a2) node[btdl@color@primary,pos=0.5] {$=$};
    \path (a2) -- (a3) node[btdl@color@primary,pos=0.5] {$=$};
   \node[below=.15cm] at (current bounding box.south) {Reproducible};

\end{tikzpicture}
\end{layout-imageonly}
\end{frame}


\subsection{This is an example, 2}

\begin{frame}{Mega-Example 1}
    \begin{layout-imageonly}
        \begin{tikzpicture}
        \node at(0.75,0.25) {\typesetBox{>\,500\,million users}{XcodeGhost}{2015}{dirac}};
        \node at(7,2) {\typesetBox{\textasciitilde\,18\,000 customers}{SolarWinds}{2020}{dirac}};
        \node at(3.33,-2) {\typesetBox{>\,100\,000 users}{Win32/Induc}{2009}{dirac}};
        \end{tikzpicture}
    \end{layout-imageonly}
\end{frame}

\begin{frame}{Mega-Example 2}
    World \cite{knuth-web}!
\end{frame}

\section{Talk the talker}
\subsection{This is an example}
\SidebarReset
\SidebarCite{dirac}
\begin{frame}[fragile]{Mega-Example 1}
    \begin{lstlisting}[language = C]
    //This is a comment
    void function(){
        lolz("Hello World!");
    }
    \end{lstlisting}
\end{frame}


\end{document}
